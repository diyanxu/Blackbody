\documentclass{article}
\usepackage[utf8]{inputenc}
\usepackage{amsmath}
\usepackage{graphicx}
\usepackage{circuitikz}
\usepackage{caption}
\usepackage{subcaption}
\hbadness=99999

\title{Investigation of the Wein's displacement law and the Stefan-Boltzmann law}
\author{Di Yan Xu, Rickie Li}
\date{\today}

\begin{document}

\maketitle

\newpage

\section{Introduction}
The purpose of this experiment is to test both the Wein's displacement law and
the Stefan-Boltzmann law. The Wein's displacement law is a law used to determine
the shift in temperature due to shift in wavelength or vice-versa. The theory
behind this is because for blackbodies, as the temperature increases, the peak
of the intensity of blackbody radiation shifts to lower wavelengths. The relation
behind the shift in temperature to wavelength can be seen as
\begin{align*}
    \lambda_{\max}T = 2.898\times10^{-3}m\cdot K
\end{align*}
Where $\lambda_{\max}$ is the wavelength at the peak, and $T$ is the absolute max
temperatureof the emitting object.\\
The Stefan-Boltzmann law is the law used to determine the total intensity radiated
within all wavelength, mathmatically this is done by taking the area under the
graph of Planck's radiation law. Planck's radiation law is the following
\begin{align*}
    I(\lambda, T) &= \frac{2\pi c^2h}{\lambda^5}\frac{1}{e^{\frac{hc}{\lambda kT}} - 1}
\end{align*}
Where $h = 60626\times 10^{-34}J\cdot s$ is the Planck's constant, $k = 1.38
\times 10^{-23}m^2\cdot kg\cdot s^{-2}\cdot K^{-1}$ is the Boltzmann constant,
$T$ is the absolute temperature, and $\lambda$ is the wavelength.\\
To get the area under the curve for this, we would integrate over $\lambda$ to get
\begin{align*}
    \int_0^{\infty}I(\lambda, T) d\lambda &= \int_0^{\infty}
    \frac{2\pi c^2h}{\lambda^5}\frac{1}{e^{\frac{hc}{\lambda kT}} - 1}d\lambda\\
    &= 2\pi hc^2\int_0^{\infty}\frac{d\lambda}{\lambda^5}\frac{1}{e^{\frac{hc}{\lambda kT}} - 1}\\
    &= \sigma T^4
\end{align*}
Where $\sigma = 5.67 \times 10^{-8}W\cdot m^{-2}\cdot K^{-4}$ is the Stefan-Boltzmann
constant.

\newpage
\section{Methods}
Lazy

\newpage

\section{Analysis}
\subsubsection*{Calculate the experimental peak wavelength values $\lambda_{\exp}$}
We started by collecting data on the peaks with a set voltage $V = 5V$. Then
using the angle at which the curve hit it's peak, we compute the index of refraction
using the equation
\begin{align*}
    n = \sqrt{\bigg(\frac{2}{\sqrt{3}} \sin\theta + \frac{1}{2}\bigg)^2 + \frac{3}{4}}
\end{align*}
where $\theta = \theta_0 - \theta_{peak}$. We then converted all the index of
refraction values into wavelength values using the Cauchy equation for the relation
of index of refraction and wavelength.
\begin{align*}
    n(\lambda) &= \frac{A}{\lambda^2} + B\\
    \lambda &= \sqrt{\frac{A}{n - B}}
\end{align*}
Where the $A$ and $B$ values for the prism are $A = 13900$ and $B = 1.689$. 
Completing all the steps above, we have the calculated values of $\lambda_{\exp}$
in table 1.

\subsubsection*{Calculate temperature values $T$ using voltage and current}
We were able to calculate the temperature value using voltage and current with
the following equation
\begin{align*}
    T = 300K + \frac{\frac{V / I}{R_0} - 1}{\alpha_0}
\end{align*}
Using the above equation we calculated temperature values $T$ in Table 1.

%TODO: gonna insert data later
\begin{center}
    \begin{tabular}{ c | c | c | c | c}
     $\theta_{peak}$ & Current ($A$) & Index & $\lambda_{\max}$ & $T$\\
     \hline 
     cell4 & cell5 & cell6 &  &\\  
     cell7 & cell8 & cell9 &  &
    \end{tabular}
    \captionof{table}{Constant Voltage $V = 5V$}
\end{center}

\newpage

\subsubsection*{Use data from Wien's Law and calculate the average value of $T\times \lambda_{\exp}$.
How does it compare with Wien's Displacement Law}
Using data calculated from Table 1, we can calculate the average $\lambda_{\max}$
to be $\overline{\lambda_{\max}} = $, and the average $T$ is $\overline{T} = $.
The error is calcualted to be the sample standard deviation. Multiplying these
values together we get
\begin{align*}
    \text{math here with error prop}
\end{align*}

\subsubsection*{Question with S-B plots here}

\newpage
\subsubsection*{How did the colour of the bulb change with temperature? How did
the colour composition of the spectrum change with temperature? Considering the
peak wavelengths, why is a bulb's filament red at low temperatures and white at
high temperatures?}
To follow the Wien's displacement law, at lower temperatures there needs to be
longer wavelengths and vice-versa for shorter wavelength, a relation that can be
seen as 
\begin{align*}
    \lambda_{\max}\uparrow \implies T\downarrow \ \ \ \ \ \ \ \ \ 
    \lambda_{\max}\downarrow \implies T\uparrow
\end{align*}
Therefore this relation causes the colour of the bulb to change as the temperature
changes, in the spectrum of visible light the lower wavelengths appear has
all the colours of light thus appearing white, while for lower temperatures 
(higher wavelengths) would result in more red colours.

\subsubsection*{Last question all you}


\section{Conclusion}
LMAO KEKW I GOT THE RIVER PROBLEMS TO DO NOW

\end{document}